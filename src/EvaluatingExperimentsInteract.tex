\documentclass{beamer}
\usepackage{hyperref} % URLs
\title[Critical Thinking]{Evaluating experiments}
%\subtitle[short]{full}
\author{Andy J. Wills}
\date{}
\begin{document}

\frame{\titlepage}

\begin{frame}{Spurious correlations}
	\url{http://www.tylervigen.com/spurious-correlations}
\end{frame}

\begin{frame}{Depression and memory}
\begin{itemize}
\item Depression is associated with over-general memory.
\vspace{12 pt}
\item Depression causes memory problems? 
\item Memory problems cause depression?
\item Both causal directions?
\item Neither causal direction (e.g. both caused by childhood trauma). 
\vspace{12 pt}
\item It is not possible to distinguish between these accounts on the basis of correlational data.
\end{itemize}
\end{frame}

\begin{frame}{Longitudinal data does not solve this problem}
\begin{itemize}
\item Use of night lights in infancy is correlated with myopia in later life (true).
\vspace{12 pt}
\item Seems causal? Causes must precede effects. The later myopia cannot cause the earlier use of night lights.
So, night lights must be causing myopia?

\item Ban night lights? (genuinely recommended on basis on these data).
\end{itemize}
\end{frame}

\begin{frame}{Third factor explanations are still possible in longitudinal research}
\begin{itemize}
\item A third factor causes both the presence of night lights and myopia.
\item Developing myopia in later life has a genetic component. If your parents are myopic, this increase the chance you will become myopic.
\item Myopic adults, on average, favour higher levels of illumination. This drives their decision to use night lights in their baby's room. 
\item The parents' myopia causes both the presence of infant night lights and later myopia.
\item Ban night lights? Clearly, this would be ineffective.
\end{itemize}
\end{frame}

\begin{frame}{Correlation does not imply causation}
\begin{itemize}
\item Correlational research is fundamentally limited.
\item It is extremely unlikely that any two variables are completely unrelated.
\item Many correlations in psychology are very small e.g.
\begin{itemize}
\item Extroversion explaining 2\% of the variation in some other variable.
\item 2\% is detectably different from no correlation
\item but not meaningful (everything likely to be related to some degree).
\end{itemize}
\end{itemize}
\end{frame}

\begin{frame}{Determining causation through the Experimental Method}
\begin{itemize}
\item Simplest form
\begin{itemize}
\item Take two groups of people
\item Do different things to those two groups.
\item Measure something
\end{itemize}
\vspace{12 pt}
\item Independent variable - Intended difference in what we do to the two groups
\item Dependent variable - The thing we measure
\end{itemize}
\end{frame}

\begin{frame}{Example: Testing a treatment for depression}
\begin{itemize}
\item Group 1 - 6 weeks of the new therapy
\item Group 2 - Nothing.
\item Take measure of depression at end (e.g. Beck Depression Inventory).
\item Group 1 are less depressed than Group 2
\vspace{12 pt}
\item This has the potential to show that the therapy \emph{causes} a reduction in depression.
\item ...but there are other explanations.
\end{itemize}
\end{frame}

\begin{frame}{Pre-existing differences}
\begin{itemize}
\item Group 1 - 6 weeks of the new therapy
\item Group 2 - Nothing.
\vspace{12 pt}
\item What if Group 1 were happier to start with?
\vspace{12 pt}
\item Approaches to this problem
\begin{itemize}
\item Detection
\item Prevention
\end{itemize}
\end{itemize}
\end{frame}

\begin{frame}{Detection}
\begin{itemize}
\item Take pre-treatment measures
\item e.g. Measure BDI of both groups before (and after) treatment period.
\end{itemize}
\begin{tabular} {c c c}
			&	Pre	 &	Post \\
Therapy	&	25	&		5 \\
Control	&	25	&		25 \\
\end{tabular} 
\end{frame}

\begin{frame}{Prevention}
\begin{itemize}
\item Construct groups such that we eliminate pre-existing differences.
\item Matching - Take BDI measures for everyone. Allocate people to groups in such a way that the average BDI for the two groups is identical (or at least, minimized).
\item Randomization - Allocate people to groups randomly.
\item Matching versus Randomization - pros and cons.
\end{itemize}
\end{frame}

\begin{frame}{Our therapy experiment}
\begin{itemize}
\item Use large, randomized groups.
\item Take pre-treatment measures
\item Treatment caused the reduction in depression?
\end{itemize}
\vspace{12 pt}
\begin{tabular} {c c c}
			&	Pre	 &	Post \\
Therapy	&	25	&		5 \\
Control	&	25	&		25 \\
\end{tabular} 
\end{frame}

\begin{frame}{Attrition}
\begin{itemize}
\item Attrition - participants dropping out before the end of the study
\item If attrition rates vary between conditions, you may have a major problem.
\end{itemize}
\end{frame}

\begin{frame}{Example}
\begin{itemize}
\item Pre-treatement BDI scores
\end{itemize}
\begin{tabular} {c c c c c c c}
 &  &  &  &  &  & Mean \\
Therapy	& 6 & 8	& 12 & 15 & 30 & 14.2 \\
Control	& 6 & 8	& 12 & 15 & 30 & 14.2 \\
\end{tabular} 
\vspace{12 pt}
\begin{itemize}
\item The most-depressed 20\% drop out of therapy (perhaps because the therapy is quite demanding).
\item There are no drop-outs in the control condition (there's not much to drop out from).
\item Both therapy and control are inert (no effect) - post-treatment BDI equals pre-treatment BDI.
\end{itemize}
\end{frame}

\begin{frame}{Example}
\begin{itemize}
\item Pre-test BDI scores
\end{itemize}
\begin{tabular} {c c c c c c c}
 &  &  &  &  &  & Mean \\
Therapy	& 6 & 8	& 12 & 15 & 30 & 14.2 \\
Control	& 6 & 8	& 12 & 15 & 30 & 14.2 \\
\end{tabular} 
\begin{itemize}
\item Post-test BDI scores
\end{itemize}
\begin{tabular} {c c c c c c c}
 &  &  &  &  &  & Mean \\
Therapy	& 6 & 8	& 12 & 15 &  & 10.25 \\
Control	& 6 & 8	& 12 & 15 & 30 & 14.2 \\
\end{tabular} 

\vspace{12 pt}
\begin{itemize}
\item A therapy we know to be ineffective appears to have worked, due to non-random attrition.
\end{itemize}
\end{frame}

\begin{frame}{Placebo effect}
\begin{itemize}
\item Classic example
\begin{itemize}
\item Someone  has a headache
\item Give them a pill with no active ingredient
\item Tell them it's a headache tablet
\item Their headache symptoms reduce
\end{itemize}
\item Lesson - In order to assess drug effectiveness you need to test drug vs. placebo, NOT drug vs. nothing.
\end{itemize}
\end{frame}

\begin{frame}{Placebo effect in psychological therapy}
\begin{itemize}
\item Perhaps the therapy is inert?
\item The treatment group are happier because they have the expectation that what they are receiving will work.
\item Problem - a placebo pill is known to be inert; what is the equivalent in therapy?
\item There is no agreement - there's someone willing to endorse the effectiveness of almost any therapy.
\end{itemize}
\end{frame}

\begin{frame}{Placebo effect in psychological therapy}
\begin{itemize}
\item Solution - set out to show that your new therapy works better than an existing treatment (or, as well as existing treatment, if yours is better in some practical way e.g. cheaper).
\item Problem - this is seldom done.
\end{itemize}
\end{frame}

\begin{frame}{Experimenter Effects - Data analysis - Example}
\begin{itemize}
\item Diary entries as a measure of happiness.
\item Participants write about their feelings
\item Experimenter rates for level of happiness.
\item If experimenter knows which condition the participant is in, this may bias their assessment of happiness.

\end{itemize}
\end{frame}

\begin{frame}{Experimenter Effects - Data analysis  }
\begin{itemize}
\item Objective measures immune? 
\item No! - Data analysis typically involves many decisions, all open to bias. 
\item If the experimenter knows which condition the participants are in, this could bias their decisions.
\end{itemize}
\end{frame}

\begin{frame}{Blind testing}
\begin{itemize}
\item Single-blind testing - participant does not know which condition they are in. 
\begin{itemize}
\item e.g. Drug vs. placebo. Participants do not know which condition they are in. 
\end{itemize}
\item Double-blind testing - single-blind testing plus the experimenters do not know which condition is which until after they have completed their analysis.
\end{itemize}
\end{frame}

\begin{frame}{Pre-registration}
     ``The first principle is that you must not fool yourself, and you are the easiest person to fool'' - Richard Feynman.
\vspace{24pt}
  \begin{itemize}
  \item Record your hypothesis, method, and analysis plan, before you analyse the data.
\end{itemize}
\end{frame}


\begin{frame}{Difference versus no difference designs}
\begin{itemize}
\item The preferred hypothesis is that people differ in the speed with which they react to auditory and visual alarm signals.
\item The alternative theory against which this is compared is that there is no difference (nil hypothesis).
\item Problem - Experimental control is never perfect.
\item Thus - the nil hypothesis is almost certainly wrong, and detectably so if you test enough people.
\item Thus - the result of the study is known before you run it.
\item Thus - There was no point in running it.
\end{itemize}
\end{frame}

\begin{frame}{Better alternatives 1}
\begin{itemize}
\item Directional hypotheses
\begin{itemize}
\item The preferred theory is that auditory is faster.
\item The alternative theory against which this is compared is that there is no difference (nil hypothesis).
\item If you find visual faster, you have disproved your theory.
\item So, whatever the result, there was a point to running this experiment (because the theory was falsifiable).
\end{itemize}
\end{itemize}
\end{frame}

\begin{frame}{Better alternatives 2}
\begin{itemize}
\item Strong inference
\begin{itemize}
\item One well-established theory predicts that auditory is faster.
\item Another well-established theory predicts that visual is faster.
\item Whatever you find in this study, you've gained information (except in the unlikely case where the nil hypothesis was true).
\end{itemize}
\end{itemize}
\end{frame}

\begin{frame}{Evaluating an experiment}

  Mueller, P. A., \& Oppenheimer, D. M. (2014). The pen is mightier than the keyboard: Advantages of longhand over laptop note taking. \emph{Psychological science, 25}, 1159-1168.

  \vspace{12pt}
  
\begin{enumerate}
\item Find the full text of this paper on Google Scholar
\item Read from the title up to, but not including, the ``Study 2'' subheading.
\item Evaluate how good Study 1 is, using the \emph{checklist} to help remind you of what we've covered today.
\item Agree on a score, be ready to report your score, and to answer some questions.
\end{enumerate}
\end{frame}

\begin{frame}{Further reading/ watching}

  The notes for this lecture cover a number of additional relevant topics.

\end{frame}

\tiny This work is licensed under a Creative Commons
Attribution-ShareAlike 4.0 International Licence. This license does
not cover the work contained in the documents linked above. Last
update: \today


\end{document}

%%% Local Variables:
%%% mode: latex
%%% TeX-master: t
%%% End:
