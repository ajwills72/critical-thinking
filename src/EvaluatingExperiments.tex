\documentclass{beamer}
\usepackage{hyperref} % URLs
\title[Critical Thinking]{Evaluating experiments}
%\subtitle[short]{full}
\author{Andy J. Wills}
\date{\today}
\begin{document}

\frame{\titlepage}

\begin{frame}{Spurious correlations}
	\url{http://www.tylervigen.com/spurious-correlations}
\end{frame}


\begin{frame}{Piracy and global warming}
\begin{itemize}
\item Average global temperatures have risen over the last 300 years
\item Number of cases of sea piracy have dropped over the same period
\item Global warming and piracy are (negatively) correlated
\vspace{12 pt}
\end{itemize}
\end{frame}



\begin{frame}{Piracy and global warming}
\begin{itemize}
\item Average global temperatures have risen over the last 300 years
\item Number of cases of sea piracy have dropped over the same period
\item Global warming and piracy are (negatively) correlated
\vspace{12 pt}
\item Global warming caused by the absence of sea pirates? 
\item Would an effective way of reducing global warming be to encourage piracy?
\vspace{12 pt}
\item Absence of sea pirates caused by global warming? 
\item Would an effective way of reducing piracy be to increase our carbon emissions?
\vspace{12 pt}
\item Association does not demonstrate causation. 
\end{itemize}
\end{frame}



\begin{frame}{Depression and memory}
\begin{itemize}
\item Depression is associated with over-general memory.
\vspace{12 pt}
\item Depression causes memory problems? 
\item Memory problems cause depression?
\item Both causal directions?
\item Neither causal direction (e.g. both caused by childhood trauma). 
\vspace{12 pt}
\item It is not possible to distinguish between these accounts on the basis of correlational data.
\end{itemize}
\end{frame}




\begin{frame}{Longitudinal data does not solve this problem}
\begin{itemize}
\item Use of night lights in infancy is correlated with myopia in later life (true).
\vspace{12 pt}
\item Seems causal? Causes must precede effects. The later myopia cannot cause the earlier use of night lights.
So, night lights must be causing myopia?

\item Ban night lights? (genuinely recommended on basis on these data).
\end{itemize}
\end{frame}



\begin{frame}{Third factor explanations are still possible in longitudinal research}
\begin{itemize}
\item A third factor causes both the presence of night lights and myopia.
\item Developing myopia in later life has a genetic component. If your parents are myopic, this increase the chance you will become myopic.
\item Myopic adults, on average, favour higher levels of illumination. This drives their decision to use night lights in their baby's room. 
\item The parents' myopia causes both the presence of infant night lights and later myopia.
\item Ban night lights? Clearly, this would be ineffective.
\end{itemize}
\end{frame}



\begin{frame}{Correlation does not imply causation}
\begin{itemize}
\item Correlational research is fundamentally limited.
\item It is extremely unlikely that any two variables are completely unrelated.
\item Many correlations in psychology are very small e.g.
\begin{itemize}
\item Extroversion explaining 2\% of the variation in some other variable.
\item 2\% is detectably different from no correlation
\item but not meaningful (everything likely to be related to some degree).
\end{itemize}
\item Partial solution - set a minimum effect size (e.g. ignoring anything below 20\% of variation).
\item Still doesn't establish causation.
\end{itemize}
\end{frame}



\begin{frame}{Determining causation through the Experimental Method}
\begin{itemize}
\item Simplest form
\begin{itemize}
\item Take two groups of people
\item Do different things to those two groups.
\item Measure something
\end{itemize}
\vspace{12 pt}
\item Independent variable - Intended difference in what we do to the two groups
\item Dependent variable - The thing we measure
\end{itemize}
\end{frame}


\begin{frame}{Example: Testing a treatment for depression}
\begin{itemize}
\item Group 1 - 6 weeks of the new therapy
\item Group 2 - Nothing.
\item Take measure of depression at end (e.g. Beck Depression Inventory).
\item Group 1 are less depressed than Group 2
\vspace{12 pt}
\item This has the potential to show that the therapy \emph{causes} a reduction in depression.
\item Q: Other possibilities?
\end{itemize}
\end{frame}




\begin{frame}{Pre-existing differences}
\begin{itemize}
\item Group 1 - 6 weeks of the new therapy
\item Group 2 - Nothing.
\vspace{12 pt}
\item What if Group 1 were happier to start with?
\vspace{12 pt}
\item Approaches to this problem
\begin{itemize}
\item Detection
\item Prevention
\end{itemize}
\end{itemize}
\end{frame}


\begin{frame}{Detection}
\begin{itemize}
\item Take pre-treatment measures
\item e.g. Measure BDI of both groups before (and after) treatment period.
\end{itemize}
\begin{tabular} {c c c}
			&	Pre	 &	Post \\
Therapy	&	25	&		5 \\
Control	&	25	&		25 \\
\end{tabular} 
\end{frame}



\begin{frame}{Prevention}
\begin{itemize}
\item Construct groups such that we eliminate pre-existing differences.
\item Matching - Take BDI measures for everyone. Allocate people to groups in such a way that the average BDI for the two groups is identical (or at least, minimized).
\item Randomization - Allocate people to groups randomly.
\item Matching versus Randomization - pros and cons.
\end{itemize}
\end{frame}



\begin{frame}{Quasi-experimental design}
\begin{itemize}
\item Sometimes, prevention is not possible - we have the groups we are given.
\item This is described as a \emph{quasi-experimental} design.
\item ``Queasy experimental'' - Don Campbell.
\item More neutrally - nonequivalent groups design (one type of quasi-experimental design)
\item Detection of pre-existing differences is particularly important in NEGD.
\end{itemize}
\end{frame}


\begin{frame}{Our therapy experiment}
\begin{itemize}
\item Use large, randomized groups.
\item Take pre-treatment measures
\item Treatment caused the reduction in depression?
\end{itemize}
\vspace{12 pt}
\begin{tabular} {c c c}
			&	Pre	 &	Post \\
Therapy	&	25	&		5 \\
Control	&	25	&		25 \\
\end{tabular} 
\end{frame}

\begin{frame}{Confounding variables}
\begin{itemize}
\item Any variable, other than the one you are attempting to study, that varies between conditions, and which could potentially have led to the effect you observe.
\end{itemize}
\end{frame}

\begin{frame}{Attrition}
\begin{itemize}
\item Attrition - participants dropping out before the end of the study
\item If attrition rates vary between conditions, you may have a major problem.
\end{itemize}
\end{frame}

\begin{frame}{Example}
\begin{itemize}
\item Pre-treatement BDI scores
\end{itemize}
\begin{tabular} {c c c c c c c}
 &  &  &  &  &  & Mean \\
Therapy	& 6 & 8	& 12 & 15 & 30 & 14.2 \\
Control	& 6 & 8	& 12 & 15 & 30 & 14.2 \\
\end{tabular} 
\vspace{12 pt}
\begin{itemize}
\item The most-depressed 20\% drop out of therapy (perhaps because the therapy is quite demanding).
\item There are no drop-outs in the control condition (there's not much to drop out from).
\item Both therapy and control are inert (no effect) - post-treatment BDI equals pre-treatment BDI.
\end{itemize}
\end{frame}

\begin{frame}{Example}
\begin{itemize}
\item Pre-test BDI scores
\end{itemize}
\begin{tabular} {c c c c c c c}
 &  &  &  &  &  & Mean \\
Therapy	& 6 & 8	& 12 & 15 & 30 & 14.2 \\
Control	& 6 & 8	& 12 & 15 & 30 & 14.2 \\
\end{tabular} 
\begin{itemize}
\item Post-test BDI scores
\end{itemize}
\begin{tabular} {c c c c c c c}
 &  &  &  &  &  & Mean \\
Therapy	& 6 & 8	& 12 & 15 &  & 10.25 \\
Control	& 6 & 8	& 12 & 15 & 30 & 14.2 \\
\end{tabular} 

\vspace{12 pt}
\begin{itemize}
\item A therapy we know to be ineffective appears to have worked, due to non-random differential attrition.
\end{itemize}
\end{frame}


\begin{frame}{Hawthorne Effect}
\begin{itemize}
\item 5-year study in the Hawthorne Works near Chicago.
\item Productivity measured in secret before start of study.
\item Group of workers chosen for study on how various factors affected productivity.
\item Procedure involved discussing changes with the workers and at times using their suggestions.
\end{itemize}
\end{frame}

\begin{frame}{Working Day study}
\begin{itemize}
\item Discussion with workers
\item Shorten working day by 30 minutes
\item Productivity goes up
\item Discussion with workers
\item Shorten working day further.
\item Productivity goes up again.
\item So far, so simple - some kind of fatigue/boredom effect?
\end{itemize}
\end{frame}

\begin{frame}{Working Day study}
\begin{itemize}
\item Discussion with workers
\item Go back to original day length
\item Productivity goes up again!
\item What's going on here?
\end{itemize}
\end{frame}

\begin{frame}{Hawthorne Effect}
\begin{itemize}

\item \textbf{Hawthorne Effect}: Where participants modify their behaviour as a result of the attention they receive from the experimenter.

\item Humans are complex systems, and what the experimenter thinks she is changing is not always the only thing (or even the most important) thing that is changing from the participant's perspective.

\end{itemize}
\end{frame}

\begin{frame}{Placebo effect}
\begin{itemize}
\item Classic example
\begin{itemize}
\item Someone  has a headache
\item Give them a pill with no active ingredient
\item Tell them it's a headache tablet
\item Their headache symptoms reduce
\end{itemize}
\item Lesson - In order to assess drug effectiveness you need to test drug vs. placebo, NOT drug vs. nothing.
\item The placebo effect typically accounts for some, but not all, of a drug's effectiveness.
\item In the case of anti-depressant medication, the effect seems to be almost entirely placebo.
\end{itemize}
\end{frame}

\begin{frame}{Placebo effect in psychological therapy}
\begin{itemize}
\item Perhaps the therapy is inert?
\item The treatment group are happier because they have the expectation that what they are receiving will work.
\item Problem - a placebo pill is known to be inert; what is the equivalent in therapy?
\item There is no agreement - there's someone willing to endorse the effectiveness of almost any therapy.
\end{itemize}
\end{frame}

\begin{frame}{Placebo effect in psychological therapy}
\begin{itemize}
\item Solution - set out to show that your new therapy works better than an existing treatment (or, as well as existing treatment, if yours is better in some practical way e.g. cheaper).
\item Problem - this is seldom done.
\item Worse problem - where it is done, treatments seldom differ. 
\item Example - Posting a pamphlet on CBT as effective as 6 weeks of 1-to-1 sessions with therapist
\end{itemize}
\end{frame}



\begin{frame}{Demand characteristics}
\begin{itemize}
\item Participants' responses may be affected by a desire to comply with what they think the experimenter wants to see.
\end{itemize}
\end{frame}



\begin{frame}{Example - Evaluative conditioning}
\begin{itemize}
\item Pairing something neutral with something people already like increases their liking of the neutral item.
\item Applied in advertising
\item Coke can paired with beautiful smiling people.
\item Department store paired with heart-warming story of cross-species friendship 
\end{itemize}
\end{frame}

\begin{frame}{Evaluative conditioning - Experimental demonstration}
\begin{itemize}
\item Show picture of soft-drink can.
\item Pair repeatedly with something positive.
\item Liking ratings go up in this treatment group...
\item .. but not in a control group where the can and smiles are both presented, but in an unpaired fashion. 
\item Evidence for evaluative conditioning?
\end{itemize}
\end{frame}

\begin{frame}{Evaluative conditioning - Alternative explanation}
\begin{itemize}
\item Participant thinks - ``what's going on here? The experimenter is showing me this coke can and then smiley faces. I think they expect me to like coke more as a result. I wouldn't want to disappoint them so sure, let's give it a higher rating than I did last time''.
\end{itemize}
\end{frame}

\begin{frame}{Further reading/ watching}

Only lecture content on these topics is examinable.

\begin{itemize}
\item \url{https://www.youtube.com/watch?v=NW2EmATcb6o}
\item \url{http://www.youtube.com/watch?v=ZgXfWmgA9NE} (don't watch this one if you are easily upset or offended)	
\item  \url{http://en.wikipedia.org/wiki/Hawthorne_effect}
\item  \url{http://www.socialresearchmethods.net/kb/quasiexp.php}
\end{itemize}

\tiny This work is licensed under a Creative Commons
Attribution-ShareAlike 4.0 International Licence. This license does
not cover the work contained in the documents linked above. Last
update: \today

\end{frame}

\end{document}

%%% Local Variables:
%%% mode: latex
%%% TeX-master: t
%%% End:
