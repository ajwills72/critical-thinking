\documentclass{beamer}
\usepackage{amssymb} % Therefore symbol
\usepackage{hyperref} % URLs
\title[Critical Thinking]{Evaluating arguments}
\author{Andy J. Wills}
%\subtitle[short]{full}
\date{}

\begin{document}

\frame{\titlepage}

\begin{frame}{Last week's workshops}
  \begin{itemize}
  \item Horse-race game: Normal distribution
  \item Exam-hall bingo: Sample size and statistical power
  \item Shove ha'penny: Regression to the mean
  \item Good and evil: Illusory correlation
  \end{itemize}
\end{frame}

\begin{frame}{Some arguments}
\begin{itemize}
\item Rate on a 1-6 scale (1 = Strongly agree; 6 = strongly disagree)
\item ``The gap in salary between men and women generally disappears when they are employed in the same position''
\item ``Students who drink alcohol whilst at university are more likely to become alcoholics in later life''
\item ``Religious people are generally more honest than non-religious people''
\end{itemize}
\end{frame}

\begin{frame}{Some questions}
\begin{itemize}
\item ``The gap in salary between men and women generally disappears when they are employed in the same position''
\item Would you have given the same answer if you were of the opposite gender?
\item ``Students who drink alcohol whilst at university are more likely to become alcoholics in later life''
\item Would you have given the same answer if you were a drinker / non-drinker (delete as appropriate) ?
\item ``Religious people are generally more honest than non-religious people''
\item Would you have given the same answer if you were a believer / non-believer (delete as appropriate) ?
\end{itemize}
\end{frame}

\begin{frame}{Some answers (Stanovich \& West, 2006)}
\begin{itemize}
\item ``The gap in salary between men and women generally disappears when they are employed in the same position''
\item Males: 3.1, Females: 2.6, $p<.001$
\item ``Students who drink alcohol whilst at university are more likely to become alcoholics in later life''
\item Drinkers: 2.7, Non-drinkers" 3.5, $p<.001$
\item ``Religious people are generally more honest than non-religious people''
\item Believers: 3.3, Non-believers: 2.6, $p<.001$
\end{itemize}
\end{frame}

\begin{frame}{My-side bias}
\begin{itemize}
\item This is my-side bias - widespread and not always appreciated.
\item Another example...
\end{itemize}
\end{frame}

\begin{frame}{Perceived media bias (Vallone, Ross \& Lepper, 1985)}
\begin{itemize}
\item Pro-Israeli and pro-Arab students were shown 6 news segments of the 1982 killing of civilian refugees in Lebanon.
\item Then both rated media bias in the same segments on a 1-9 scale.
\item 1 (Anti-Israel) - 5 (neutral) - 9 (Pro-Israel)
\item Pro-Israeli students:  3.0
\item Pro-Arab students: 6.5
\item Both groups believed the media reports were biased against their own viewpoint!
\end{itemize}
\end{frame}

\begin{frame}{Perkins et al. (1991)}
Asked participants about issues of social and political significance, e.g. ``Would providing more money for state schools significantly improve the quality of teaching and learning?''

Participants had to:
\begin{enumerate}
\item Make an immediate decision.
\item Give their considered opinion, stating reasons for their opinion.
\item ``Scaffolding'' - Prompted by experimenter to consider all aspects, to look for arguments on both sides, to evaluate objectively.
\end{enumerate}
\end{frame}

\begin{frame}{Perkins et al. (1991)}
\begin{itemize}
\item Considered opinion: 3 my-side reasons, 1 other-side reason.
\item Very little improvement with age, experience, or level of formal education.
\item Scaffolded: 6 on both sides
\end{itemize}
\end{frame}

\begin{frame}{My-side bias answers}
\begin{itemize}
\item My-side bias seems to have something to do with failing to consider counter-arguments.
\item One way to combat my-side bias is to ``scaffold yourself'' with the aid of pen and paper. Write down the possibilities, write down the counter-arguments, be suspicious if the list for your preferred answer is much longer than the list for your non-preferred answer.
\end{itemize}
\end{frame}

\begin{frame}{Weak arguments}
Weak arguments are statements intended to support a conclusion, but which do not do so (or do so only very weakly). 
\end{frame}

\begin{frame}{Non-sequitur}
\begin{itemize}
\item Latin - ``it does not follow''
\item Any invalid argument is a non-sequitur
\begin{itemize}
	\item All dogs are mammals.
	\item All dogs bark.
	\item Therefore, all mammals bark.
\end{itemize}
\end{itemize}
\end{frame}

\begin{frame}{Equivocation}
\begin{itemize}
\item A feather is light.
\item What is light cannot be dark.
\item Therefore, a feather cannot be dark.

The equivocation fallacy is the typically the product of a word having more than one meaning, and that meaning not being specified. 

\item ``Do women need to worry about man-eating sharks?''
\end{itemize}
\end{frame}

\begin{frame}{Ad hominem}
\begin{itemize}
\item Latin - ``at the man''.
\item Attacking the person instead of the argument.\textbf{Being a hypocrite does not make your claim false}.
\item ``How can you argue for vegetarianism when you wear leather shoes?''
\item ``Anyone over the age of thirty who believes in Socialism has no brain''.
\item Highlighting weaknesses of character in your opponent (imagined or real) does not prove their position is wrong.
\item ``Dr. Bottle tells us not to drink and drive, but I know he always has a few pints before driving home [so it's safe to ignore his advice]''.

\end{itemize}
\end{frame}

\begin{frame}{Appeal to force}
\begin{itemize}
\item Supporting one's argument by force or the threat of force.
\item From the 14th century - ``The clinching proof of my reasoning is that I will cut anyone who argues further into dogmeat'' (de Tourneville, 1350).
\item  From the Reagan era (1980s) - ``The President continues to have confidence in the Attorney General ... and you ought to have confidence in the Attorney General, because we work for the President and that is the way things are ... if anyone has a different view on [the Attorney General] he can tell me about it because we are going to have to discuss your status''
\end{itemize}
\end{frame}

\begin{frame}{Begging the question}
\begin{itemize}
\item In common usage ``begs the question'' means ``raises the question'', ``evades the questions'', or ``ignores the question''. 
\item In critical thinking, it has a narrower, somewhat different definition.
\item Begging the question - Assuming the truth of a conclusion in order to provide support for it.
\item For example, ``Opium induces sleep because it has a soporific quality'' (``soporific'' means sleep-inducing).
\item Thus, opium induces sleep because it induces sleep.
\item Begging the question sometimes convinces because we don't notice the synonym.
\end{itemize}
\end{frame}

\begin{frame}{Argument from ignorance}
\begin{itemize}
\item ``One cannot prove that God does not exist'' (when used to support the existence of God).
\item ``One cannot prove this new teaching method will make things worse'' (when used to support the conclusion that it will make things better). 
\item Fallacy: If a proposition has not been disproven, then it cannot be considered false and must therefore be considered true.
\item Fallacy: If a proposition has not been proven, then it cannot be considered true and must therefore be considered false.
\item Both are fallacious because the limits of one's understanding do not change what is true.
\end{itemize}
\end{frame}



\begin{frame}{Infallible Flowchart of Argument Evaluation}
\begin{enumerate}
\item Identify the conclusion.
\item Identify the premise or premises.
\item Identify the relationship between the premise(s) and conclusion.
\begin{itemize}
\item Independent?
\item Conjoint?
\item Causal chain?
\end{itemize}
\item Do the premises support the conclusion?
\begin{itemize}
\item Logical deduction?
\item Reasonable inference?
\end{itemize}
\item Are the premises true?
\end{enumerate}
\end{frame}


\begin{frame}{Evaluating an argument}
``Contrary to what people think, a positive test for HIV is not necessarily a death sentence. For one thing, the time from the development of antibodies to clinical symptoms averages nearly ten years. For another, many reports are now suggesting that a significant number of people who test positive may never develop clinical AIDS''.
\end{frame}

% So, what do we do first (PAUSE)... Identify the conclusion! Right, what's the conclusion here?

\begin{frame}{1. Identify the conclusion}
``Contrary to what people think, \textbf{a positive test for HIV is not necessarily a death sentence}. For one thing, the time from the development of antibodies to clinical symptoms averages nearly ten years. For another, many reports are now suggesting that a significant number of people who test positive may never develop clinical AIDS''.
\end{frame}

% Good.
% Now what is the premise, or what are the premises?

\begin{frame}{2. Identify the premise(s)}
``Contrary to what people think, \textbf{a positive test for HIV is not necessarily a death sentence}. For one thing, \emph{the time from the development of antibodies to clinical symptoms averages nearly ten years}. For another, many reports are now suggesting that \emph{a significant number of people who test positive may never develop clinical AIDS}''.
\end{frame}

% OK, good, there are two premises. 

% OK, what now? PAUSE. Right, identify the relationship between the premises and the conclusion. What were the choices? Right, independent, cojoint, or causal chain. Which of these options best represents the argument here?  Right, it's independent - they provide two two different reasons why the authors wants us to believe his conclusion. 

% Some find it helpful to represent this relationship graphically. Doing this is called producing an argument diagram. The argument diagram for the current argument is shown in the next slide. We'll return to other relationships later in the lecture.

\begin{frame}{3. Identify relationship between premises and conclusion}
	``Contrary to what people think, \textbf{a positive test for HIV is not necessarily a death sentence}. For one thing, \emph{the time from the development of antibodies to clinical symptoms averages nearly ten years}. For another, many reports are now suggesting that \emph{a significant number of people who test positive may never develop clinical AIDS}''.
	
\vspace{12pt}
Independent.
\end{frame}



\begin{frame}{4. Do the premises support the conclusion?}

	``Contrary to what people think, \textbf{a positive test for HIV is not necessarily a death sentence}. For one thing, \emph{the time from the development of antibodies to clinical symptoms averages nearly ten years}. For another, many reports are now suggesting that \emph{a significant number of people who test positive may never develop clinical AIDS}''.

\begin{itemize}
\item Difficult to say - the conclusion is a bit vague to evaluate the extent to which it is supported by the first premise.
\item The second premise seems more directly supportive of the conclusion.
\end{itemize}
\end{frame}


% Now, we get to the less formulaic part. Do the premises support the conclusion? This is quite difficult to say, and one of the reasons it's difficult is because the conclusion is somewhat vague. What does the phrase ``not necessarily a death sentence'' mean? Life is a death sentence - we all die - and in the non-metaphorical sense of the phrase, there is often a delay or years or even decades between the announcement of a death sentence and it being carried out. This makes it hard to see the first premise as supporting the conclusion - the author would need to make a more specific conclusion in order to use this premise appropriately. 

% The relationship of the second premise to the conclusion is somewhat clearer. AIDS is a cause of death, being HIV+ is a symptom. Being HIV+ does not necessarily lead to AIDS and thus does not necessarily lead to death. This is fairly close to what the author seems to wish to conclude, although again the conclusion could be clearer. 



\begin{frame}{5. Are the premises true?}

	``Contrary to what people think, \textbf{a positive test for HIV is not necessarily a death sentence}. For one thing, \emph{the time from the development of antibodies to clinical symptoms averages nearly ten years}. For another, many reports are now suggesting that \emph{a significant number of people who test positive may never develop clinical AIDS}''.

\begin{itemize}
\item First premise - Mean incubation time in young adults is 10 years (Bacchetti \& Moss, 1989).
\item Second premise:
\begin{itemize}
\item ``Some people who test positive for HIV never develop AIDS''
\item False alarms?
\item ``Some people who are HIV+ never develop AIDS''
\item Death by other causes?
\item Presence of ``long-term non-progressors'' (about 1 in 500 HIV+ are still asymptomatic after 12 years)
\end{itemize}
\end{itemize}
\end{frame}


% Finally, are crucially, are the premises true? This is the legwork of science, and its interconnectedness. Evaluation of the truth of these premises depends on other arguments, in papers, and on the soundness of those arguments. As this is a lecture on critical thinking, not on HIV and AIDS, I won't go into detail, but notice the following.

% First premise - is relatively straight forward and testable. One well-cited study estimates mean incubation time of HIV to be 10 years in young adults (Bacchetti \& Moss, 1989) so, on firs glance, the premise seems not unreasonable. 

% Second premise - ``suggesting'', ``may'' and ``significant'' are hedges. They reduce the clarity of the premise without changing its essential character. If the second premise is to support the conclusion, we have to consider it as the statement, ``some people who test positive for HIV never develop AIDS''. Put this way, the premise is hard to evaluate. There is one sense in which it is true - no test is perfect, so some people who test positive for HIV are not, in fact, HIV+. But this is unlikely to be what the author intended. If so, the statement is ``some people who are HIV+ never develop AIDS''. Again, there is a fairly trivial sense in which this is true. For example, someone who commits suicide on the news they are HIV+ is never going to develop AIDS. As a matter of principle, then, it is going to be very difficult to discount the possibility that AIDS would have developed if the patient had not died of some other cause first. 

% What the author may be referring to is the presence of what the field describes as ``long-term non-progressors'' - people who are still asymptomatic after 12 years. Such cases do exist, and their prevalence is hard to estimate, but they are certainly very rare (one estimate is 1 in 500 cases of HIV+). 

% http://www.niaid.nih.gov/volunteer/hivlongterm/Pages/default.aspx

\begin{frame}{Evaluating the argument}

	``Contrary to what people think, \textbf{a positive test for HIV is not necessarily a death sentence}. For one thing, \emph{the time from the development of antibodies to clinical symptoms averages nearly ten years}. For another, many reports are now suggesting that \emph{a significant number of people who test positive may never develop clinical AIDS}''.
	

\begin{itemize}
\item The first premise seems to be supported by scientific evidence, but the author's conclusion is too vague to be supported by the first premise.
\item The second premise supports the conclusion, but it is too vague to be clearly evaluated. If the author refers to long-term non-progressors, then these do exist, but it seems too much of a leap from an approximately 1 in 500 chance of not developing symptoms for more than 12 years to ``a significant number of people who test positive never developing AIDS''. 
\end{itemize}
\end{frame}

% Now we've got to the end of that specific example, I'm going to back track slightly to identifying the relationship between premises and the conclusion. In the above example, the premises were independent:


% Another example

\begin{frame}{Fox hunting}

``It is wrong to cause unnecessary suffering to an animal. Fox hunting causes unnecessary suffering to the fox. It is therefore wrong to hunt foxes''.

\end{frame}

% OK, what's the conclusion? The premises? Correct. Now what's the relationship between the premises and the conclusion? Independent? Cojoint? Causal chain? 

% Right, it's cojoint. Both statements have to be correct for the conclusion to be supported. If it was OK to cause unnecessary suffering, OR if it were untrue that fox hunting causes unnecessary suffering, then the conclusion would not be supported. 

\begin{frame}{Steps 1 to 3}

``\emph{It is wrong to cause unnecessary suffering to an animal}. \emph{Fox hunting causes unnecessary suffering to the fox}. \textbf{It is therefore wrong to hunt foxes}''.

\vspace{12pt}
Co-joint.

\end{frame}

% As another illustration of the next point - Do the premises support the conclusion? In what way - logical deduction or reasonable inference? 

% In this case, it's quite complex.

\begin{frame}{4. Do the premises support the conclusion?}

``\emph{It is wrong to cause unnecessary suffering to an animal}. \emph{Fox hunting causes unnecessary suffering to the fox}. \textbf{It is therefore wrong to hunt foxes}''.

\begin{itemize}
	\item Let's assume the conclusion is meant to be absolute - all fox hunting is wrong.
	\item If both premises are absolute, it's a case of logical deduction - the conclusion follows directly from the premises.
	\begin{itemize}
		\item All [causing-unnecessary-suffering] is [wrong].
		\item All [fox-hunting] is [causing-unnecessary-suffering].
		\item Therefore, all [fox-hunting] is [wrong]
	\end{itemize}
	\item If either of the premises is conditional (``fox hunting sometimes causes unnecessary suffering''), or if the conclusion is conditional, then it gets more complex. In many cases, it will be an inference rather than a deduction. 
\end{itemize}
\end{frame}

\begin{frame}{5. Are the premises true?}

See some further discussion in the notes.
	
\end{frame}


% In the interests of time, I'll leave this example here. In order to fully evaluate this argument, you would need to go on and investigate whether the premises are true. The first premise is not really a scientific claim because it is prescriptive rather than descriptive, as covered in the last lecture. The second premise is too vague to investigate, because of the term ``unnecessary''. What defines whether an event is unnecessary? However, the related claim ``fox hunting causes suffering'' is a claim for which the evidence can potentially examined be scientifically --- if one were satisfied that animal suffering is something that can be measured (which it probably is).



% One further important concept - Enthymemes. This refers to a situation, undesirable but quite common even in scientific writing, where one or more of the premises are missing and have to be inferred by the reader. For example:

\begin{frame}{Enthymemes}

``It is biologically natural for humans to eat animal flesh. Therefore, it is morally permissible for humans to eat animal flesh''.

\begin{itemize}
\item Missing premise: ``Whatever is biologically natural for humans is morally permissible''.
\end{itemize}

\end{frame}



\begin{frame}{Infallible Flowchart of Argument Evaluation and Construction!}
\begin{enumerate}
\item Identify the conclusion.
\item Identify the premise or premises.
\item Identify the relationship between the premise(s) and conclusion.
\begin{itemize}
\item Independent?
\item Co-joint?
\item Causal chain?
\end{itemize}
\item Do the premises support the conclusion?
\begin{itemize}
\item Logical deduction?
\item Reasonable inference?
\end{itemize}
\item Are the premises true?
\end{enumerate}

\vspace{12pt}

\tiny
This work is licensed under a Creative Commons Attribution-ShareAlike
4.0 International Licence. Last update: \today

\end{frame}

\end{document}

%%% Local Variables:
%%% mode: latex
%%% TeX-master: t
%%% End:
