\documentclass{beamer}
\usepackage{amssymb} % Therefore symbol
\usepackage{hyperref} % URLs
\title[Critical Thinking]{Evaluating arguments}
\author{Andy J. Wills}
%\subtitle[short]{full}
\date{}

\begin{document}

\frame{\titlepage}

\begin{frame}{Last week: Evaluating experiments}
  \begin{itemize}
  \item Poll of your answers
  \item Discussion of some key points
  \end{itemize}
\end{frame}

\begin{frame}{This week: Evaluating arguments}
  \begin{itemize}
  \item Weak arguments
  \item Infalliable flowchart of argument evaluation
  \end{itemize}
\end{frame}

\begin{frame}{Weak arguments}
Weak arguments are statements intended to support a conclusion, but which do not do so (or do so only very weakly). 
\end{frame}

\begin{frame}{Non-sequitur}
\begin{itemize}
\item Latin - ``it does not follow''
\item Any invalid argument is a non-sequitur
\begin{itemize}
	\item All dogs are mammals.
	\item All dogs bark.
	\item Therefore, all mammals bark.
\end{itemize}
\end{itemize}
\end{frame}

\begin{frame}{Equivocation}
\begin{itemize}
\item A feather is light.
\item What is light cannot be dark.
\item Therefore, a feather cannot be dark.

The equivocation fallacy is the typically the product of a word having more than one meaning, and that meaning not being specified. 

\item ``Do women need to worry about man-eating sharks?''
\end{itemize}
\end{frame}

\begin{frame}{Ad hominem}
\begin{itemize}
\item Latin - ``at the man''.
\item Attacking the person instead of the argument.\textbf{Being a hypocrite does not make your claim false}.
\item ``How can you argue for vegetarianism when you wear leather shoes?''
\item ``Anyone over the age of thirty who believes in Socialism has no brain''.
\item Highlighting weaknesses of character in your opponent (imagined or real) does not prove their position is wrong.
\item ``Dr. Bottle tells us not to drink and drive, but I know he always has a few pints before driving home [so it's safe to ignore his advice]''.

\end{itemize}
\end{frame}

\begin{frame}{Appeal to force}
\begin{itemize}
\item Supporting one's argument by force or the threat of force.
\item From the 14th century - ``The clinching proof of my reasoning is that I will cut anyone who argues further into dogmeat'' (de Tourneville, 1350).
\item  From the Reagan era (1980s) - ``The President continues to have confidence in the Attorney General ... and you ought to have confidence in the Attorney General, because we work for the President and that is the way things are ... if anyone has a different view on [the Attorney General] he can tell me about it because we are going to have to discuss your status''
\end{itemize}
\end{frame}

\begin{frame}{Begging the question}
\begin{itemize}
\item In common usage ``begs the question'' means ``raises the question'', ``evades the questions'', or ``ignores the question''. 
\item In critical thinking, it has a narrower, somewhat different definition.
\item Begging the question - Assuming the truth of a conclusion in order to provide support for it.
\item For example, ``Opium induces sleep because it has a soporific quality'' (``soporific'' means sleep-inducing).
\item Thus, opium induces sleep because it induces sleep.
\item Begging the question sometimes convinces because we don't notice the synonym.
\end{itemize}
\end{frame}

\begin{frame}{Argument from ignorance}
\begin{itemize}
\item ``One cannot prove that God does not exist'' (when used to support the existence of God).
\item ``One cannot prove this new teaching method will make things worse'' (when used to support the conclusion that it will make things better). 
\item Fallacy: If a proposition has not been disproven, then it cannot be considered false and must therefore be considered true.
\item Fallacy: If a proposition has not been proven, then it cannot be considered true and must therefore be considered false.
\item Both are fallacious because the limits of one's understanding do not change what is true.
\end{itemize}
\end{frame}

\begin{frame}{Activity: Spot the weak arguments}
\begin{itemize}
\item I'm going to prescribe you Prozac. It's an anti-depressant, so it should help with your low mood. 
  
\item Global warming is a good thing, it's always so bloody cold around here!

\item When you've studied psychology as long as I have, you'll realise that qualitative methods are inherently superior to quantitative ones.
  
\item Some immigrants are Syrian. Some Syrians are terrorists. Therefore, some immigrants are terrorists.

\item Herd immunity must be a stupid idea, look at the dumb people who support it!

\item Shape-shifting alien lizard people have replaced major world leaders. The fact they don't look like lizards just goes to show how good they are at shape shifting. 
\end{itemize}  
\end{frame}

\begin{frame}{Analyzing arguments}

\end{frame}

\begin{frame}{Infallible Flowchart of Argument Evaluation}
\begin{enumerate}
\item Identify the conclusion.
\item Identify the premise or premises.
\item Identify the relationship between the premise(s) and conclusion.
\begin{itemize}
\item Independent?
\item Conjoint?
\end{itemize}
\item Do the premises support the conclusion?
\begin{itemize}
\item Logical deduction?
\item Reasonable inference?
\end{itemize}
\item Are the premises true?
\end{enumerate}
\end{frame}

\begin{frame}{Evaluating an argument}
``Contrary to what people think, a positive test for HIV is not necessarily a death sentence. For one thing, the time from the development of antibodies to clinical symptoms averages nearly ten years. For another, many reports are now suggesting that a significant number of people who test positive may never develop clinical AIDS''.
\end{frame}

% So, what do we do first (PAUSE)... Identify the conclusion! Right, what's the conclusion here?

\begin{frame}{1. Identify the conclusion}
``Contrary to what people think, \textbf{a positive test for HIV is not necessarily a death sentence}. For one thing, the time from the development of antibodies to clinical symptoms averages nearly ten years. For another, many reports are now suggesting that a significant number of people who test positive may never develop clinical AIDS''.
\end{frame}

% Good.
% Now what is the premise, or what are the premises?

\begin{frame}{2. Identify the premise(s)}
``Contrary to what people think, \textbf{a positive test for HIV is not necessarily a death sentence}. For one thing, \emph{the time from the development of antibodies to clinical symptoms averages nearly ten years}. For another, many reports are now suggesting that \emph{a significant number of people who test positive may never develop clinical AIDS}''.
\end{frame}

% OK, good, there are two premises. 

% OK, what now? PAUSE. Right, identify the relationship between the premises and the conclusion. What were the choices? Right, independent, cojoint, or causal chain. Which of these options best represents the argument here?  Right, it's independent - they provide two two different reasons why the authors wants us to believe his conclusion. 

% Some find it helpful to represent this relationship graphically. Doing this is called producing an argument diagram. The argument diagram for the current argument is shown in the next slide. We'll return to other relationships later in the lecture.

\begin{frame}{3. Identify relationship between premises and conclusion}
	``Contrary to what people think, \textbf{a positive test for HIV is not necessarily a death sentence}. For one thing, \emph{the time from the development of antibodies to clinical symptoms averages nearly ten years}. For another, many reports are now suggesting that \emph{a significant number of people who test positive may never develop clinical AIDS}''.
	
\vspace{12pt}
Independent.
\end{frame}



\begin{frame}{4. Do the premises support the conclusion?}

	``Contrary to what people think, \textbf{a positive test for HIV is not necessarily a death sentence}. For one thing, \emph{the time from the development of antibodies to clinical symptoms averages nearly ten years}. For another, many reports are now suggesting that \emph{a significant number of people who test positive may never develop clinical AIDS}''.

\begin{itemize}
\item Difficult to say - the conclusion is a bit vague to evaluate the extent to which it is supported by the first premise.
\item The second premise seems more directly supportive of the conclusion.
\end{itemize}
\end{frame}


% Now, we get to the less formulaic part. Do the premises support the conclusion? This is quite difficult to say, and one of the reasons it's difficult is because the conclusion is somewhat vague. What does the phrase ``not necessarily a death sentence'' mean? Life is a death sentence - we all die - and in the non-metaphorical sense of the phrase, there is often a delay or years or even decades between the announcement of a death sentence and it being carried out. This makes it hard to see the first premise as supporting the conclusion - the author would need to make a more specific conclusion in order to use this premise appropriately. 

% The relationship of the second premise to the conclusion is somewhat clearer. AIDS is a cause of death, being HIV+ is a symptom. Being HIV+ does not necessarily lead to AIDS and thus does not necessarily lead to death. This is fairly close to what the author seems to wish to conclude, although again the conclusion could be clearer. 



\begin{frame}{5. Are the premises true?}

	``Contrary to what people think, \textbf{a positive test for HIV is not necessarily a death sentence}. For one thing, \emph{the time from the development of antibodies to clinical symptoms averages nearly ten years}. For another, many reports are now suggesting that \emph{a significant number of people who test positive may never develop clinical AIDS}''.

\begin{itemize}
\item First premise - Mean incubation time in young adults is 10 years (Bacchetti \& Moss, 1989).
\item Second premise:
\begin{itemize}
\item ``Some people who test positive for HIV never develop AIDS''
\item False alarms?
\item ``Some people who are HIV+ never develop AIDS''
\item Death by other causes?
\item Presence of ``long-term non-progressors'' (about 1 in 500 HIV+ are still asymptomatic after 12 years)
\end{itemize}
\end{itemize}
\end{frame}

\begin{frame}{Evaluating the argument}

	``Contrary to what people think, \textbf{a positive test for HIV is not necessarily a death sentence}. For one thing, \emph{the time from the development of antibodies to clinical symptoms averages nearly ten years}. For another, many reports are now suggesting that \emph{a significant number of people who test positive may never develop clinical AIDS}''.
	

\begin{itemize}
\item The first premise seems to be supported by scientific evidence, but the author's conclusion is too vague to be supported by the first premise.
\item The second premise supports the conclusion, but it is too vague to be clearly evaluated. If the author refers to long-term non-progressors, then these do exist, but it seems too much of a leap from an approximately 1 in 500 chance of not developing symptoms for more than 12 years to ``a significant number of people who test positive never developing AIDS''. 
\end{itemize}
\end{frame}

\begin{frame}{Activity: Analyzing a fox hunting argument}

``It is wrong to cause unnecessary suffering to an animal. Fox hunting causes unnecessary suffering to the fox. It is therefore wrong to hunt foxes''.

\end{frame}

\begin{frame}{Infallible Flowchart of Argument Evaluation and Construction!}
\begin{enumerate}
\item Identify the conclusion.
\item Identify the premise or premises.
\item Identify the relationship between the premise(s) and conclusion.
\begin{itemize}
\item Independent?
\item Co-joint?
\item Causal chain?
\end{itemize}
\item Do the premises support the conclusion?
\begin{itemize}
\item Logical deduction?
\item Reasonable inference?
\end{itemize}
\item Are the premises true?
\end{enumerate}

\vspace{12pt}

\tiny
This work is licensed under a Creative Commons Attribution-ShareAlike
4.0 International Licence. Last update: \today

\end{frame}

\begin{frame}{Further reading}

  The notes accompanying this lecture (available after the lecture has been given) cover some additional topics in evaluating arguments. The information in the notes is also examinable.

\end{frame}

\end{document}

%%% Local Variables:
%%% mode: latex
%%% TeX-master: t
%%% End:
