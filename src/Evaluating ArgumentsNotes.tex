\documentclass[12pt]{article}
\usepackage[a4paper,left=2.25cm,right=2.25cm,top=2cm,bottom=2.5cm]{geometry}
\usepackage{hyperref} % Handle web addresses
\usepackage{graphicx} % Handle graphics
\begin{document}
	\title{Evaluating arguments}
	\date{\today}
	\author{Andy J. Wills}
	\maketitle

\section{Structure of the lecture}

This lecture contained four sections, each aimed at improving your
ability to evaluate arguments --- both your own, and other
people's. In the first two sections of the lecture, I considered two
common biases in argumentation. These are biases that people often do
not realise they have, but which reliably occur unless they
deliberately takes steps to avoid them. In each case, I give advice on
how to avoid these biases in your own thinking. In the third part of
the lecture, I considered weak arguments --- how to identify them and
hence avoid being taken in by them. In the final part, I demonstrated
a systematic method for evaluating arguments.

\section{My-side bias}

The first four slides demonstrate a study by Stanovich and West
(2006). The study shows that peoples' assessment of the strength of
claims is affected by the relation of those claims to their personal
circumstances. Thus, for example, non-drinkers are more likely than
drinkers to think that drinking at university leads to alcoholism
later in life. This is called \emph{my-side bias}.

The next slide summarizes another study of my-side bias, this time in
the interpretation of news reports concerning the Lebanon War in
1982. Pro-Israeli students considered the reports to be anti-Israel,
but pro-Arab students considered the reports to be pro-Israel. So, in
this case, both sides considered the reports to be biased towards the
opposing side in the conflict. This is another example of my-side bias
--- the assessment of evidence is affected by what you already
believe.

The next two slides summarize a study by Perkins et al. (1991). The
study illustrates that my-side bias is not reduced by age, experience,
or education. However, it is reduced if participants receive a prompt
to look for arguments on both sides (``scaffolding'').

The final slide of this section summarizes what you can do to avoid
my-side bias. Basically, scaffold yourself!

\section{Illusory correlation}

Illusory correlation is the phenomenon, originally due to Hamilton and
Gifford (1976), that people will tend to perceive a correlation
between two variables when in fact one does not exist.

In Hamilton and Gifford's study, participants were presented with a
series of statements about the members of two gangs. The statements
described either desirable or undesirable behaviour. For example, ``a
member of group A visits a friend in hospital'' (desirable), or ``a
member of group B kicks a puppy'' (undesirable). Participants get 39
statements and then are asked questions such as ``How likely is it
that a member of Group A engages in desirable behaviours?''. Group A
is larger than Group B, so participants see more statements about
Group A than about Group B. The first slide shows the number of
statements they see.

The result of the study is that Group B is rated as more likely to
engage in undesirable behaviours than Group A. This is not because of
the labels ``A'' and ``B'', because these are counter-balanced
(i.e. for half the participants Group A is the minority group).

The result is striking, because there is no correlation here --- the
ratio of desirable to undesirable behaviour is the same in both
groups. The result is reliably found where (a) Group B statements are
less frequent than Group A statements, (b) desirable behaviour is more
common than undesirable behaviour, and (c) the number of observations
of Group B is quite low.

There are clear links here to the prejudice against minorities we see
in the real world, and it is interesting and important to note that
this illusion is not specific to such real world content --- both the
groups and the behaviours can be arbitrary and abstract, and you still
get this effect. It's a fundamental limitation in the human ability to
assess correlation from small samples, and a classic example of why it
is important to sit down and work out the ratios, rather than making
a judgement ``on instinct''.

\section{Weak arguments}

Weak arguments are those that do not support the arguer's conclusion
(or do so only very weakly). Weak arguments are prevalent in general
society, and even in science, because they have the \emph{appearance}
of being good arguments. I'm telling you about weak arguments so you
can detect and reject them. Do not be tempted by the \emph{dark side}
of weak arguments! They are very powerful, as they can often be used
to win arguments without having to go to the bother of considering the
evidence.

Across six slides, I summarize some of the more common weak
arguments. The examples on the slides are intended to be light-hearted
and clear, but the point is serious. Listen to media reports, read
scientific papers, and try to identify these kinds of weak
arguments. Here are some further examples

\begin{itemize}
\item ``Many migrants are from Syria, and some Syrians are terrorists,
  so some migrants are terrorists''. This is a non-seqitur.

\item ''Global warming is a good thing --- it's always so bloody cold
  around here!''. This is equivocation - it makes use of the everyday
  meaning of the word ``warming'', which is different to scientific
  meaning of the term global warming, which relates to a range of
  effects consequent to a rise in overall global temperature
  (including, potentially, a reduction in UK temperatures due to
  movement of the gulf stream).

\item ``Dr. Smith, a leading expert in sleep research, confirms that
  alarm clocks are bad for your health''. Unless one goes on to
  discuss the evidence, this is an appeal to force --- we are meant to
  be convinced by the description of this person as an expert.

\item A GP says ``I'm going to prescribe Prozac. It's an
  anti-depressant, so it should improve your mood''. By definition, an
  anti-depressant improves mood, so this statement is circular, it
  provides no evidence for the conclusion. It is begging the question.

  As an exercise, see if you can come up with your own examples of the
  weak arguments discussed in the lectures. 

\end{itemize}

\section{How to evaluate an argument}

In the final section of this lecture, I talked you through a
step-by-step method for evaluating other peoples'
arguments. Evaluating arguments is at the heart of all good scientific
writing and is central to doing well in all essays and reports
throughout your course.

\subsection{Infallible Flowchart}

I call the method the Infalliable Flowchart because it always
works. When reading a paper (or textbook) have this flowchart beside
you and go through its steps for each key claim.

You start by identifying the conclusion of the argument. If you don't
know what it is that is being claimed, you cannot evaluate that
claim. So, start with the conclusion and work backwards.

Next, you identify the premise or premises. In other words, identify
the statements that are used to support the conclusion. These are
called premises.

Next, you work out the relationship between the premise(s) and
conclusion. For example, if there are two premises, are they
independent? In other words do they each provide separate arguments
for the conclusion? Or perhaps they are conjoint? In other words, the
argument assumes that both premises have to be correct for the
conclusion to follow. Or perhaps they are casually linked? In other
words, the second premise is a conclusion drawn from the first
premise, and then the stated conclusion is drawn from the second
premise?

Next, you ask whether the premises support the conclusion. In other
words, does the conclusion follow from the reasons given. Is it a
logical deduction? A logical deduction is where the conclusion
necessarily follows, given the truth of the premise(s). Logical
deduction is quite rare. Often, an inference is being made. An
inference is a conclusion that is likely to be true given the truth of
the premises, but not inevtiably so.

Finally, you ask whether the premises are true. Even if the
conclusion logically follows from the premises, it is not a valid
conclusion unless the premises are true. In science, previous or
current research is generally used to evaluate whether the premises
are true.

\subsection{AIDS example}

We worked through an example of argument evaluation using an argument
about AIDS. We identified the conclusion and premises, and then
decided that the relationship between them was independent. The two
premises provide two different reasons why the authors wants us to
believe his conclusion. If one reason fails, the other may still
support the conclusion.

We then came to the less formulaic part --- do the premises support
the conclusion? This is quite difficult to say, and one of the reasons
it's difficult is because the conclusion is somewhat vague. What does
the phrase ``not necessarily a death sentence'' mean? Life is a death
sentence --- we all die --- and in the legally-related sense of the
phrase, there is often a delay of years or even decades between the
announcement of a death sentence and it being carried out. This makes
it hard to see the first premise as supporting the conclusion --- the
author would need to make a more specific conclusion in order to use
this premise appropriately.

The relationship of the second premise to the conclusion is somewhat
clearer. AIDS is a cause of death, being HIV+ is a symptom. If being
HIV+ does not necessarily lead to AIDS, it therefore does not
necessarily lead to death. This is fairly close to what the author
seems to wish to conclude, although again the conclusion could be
clearer.

Finally, and crucially, we asked whether the premises were true. This
is the legwork of science, and its interconnectedness. Evaluation of
the truth of these premises depends on other arguments, in papers, and
on the soundness of those arguments.

The first premise is relatively straight forward and testable. One
well-cited study estimates mean incubation time of HIV to be 10 years
in young adults (Bacchetti \& Moss, 1989) so, on first glance, the
premise seems not unreasonable.

In the second premise - ``suggesting'', ``may'' and ``significant''
are hedges. They reduce the clarity of the premise without changing
its essential character. If the second premise is to support the
conclusion, we have to consider it as the statement, ``some people who
test positive for HIV never develop AIDS''. Put this way, the premise
is hard to evaluate. There is one sense in which it is true --- no
test is perfect, so some people who test positive for HIV are not, in
fact, HIV+. But this is unlikely to be what the author intended. If
so, the statement is ``some people who are HIV+ never develop
AIDS''. Again, there is a fairly trivial sense in which this is
true. For example, someone who commits suicide on the news they are
HIV+ is never going to develop AIDS. As a matter of principle, then,
it is going to be very difficult to discount the possibility that AIDS
would have developed if the patient had not died of some other cause
first.

What the author may be referring to is the presence of what the field
describes as ``long-term non-progressors'' - people who are still
asymptomatic after 12 years. Such cases do exist, and their prevalence
is hard to estimate, but they are certainly very rare (one estimate is
1 in 500 cases of
HIV+). \url{http://www.niaid.nih.gov/volunteer/hivlongterm/Pages/default.aspx}

In summary, the first premise seems to be supported by scientific
evidence, but the author's conclusion is too vague to be supported by
the first premise.  The second premise supports the conclusion, but it
is too vague to be clearly evaluated. If the author refers to
long-term non-progressors, then these do exist, but it seems too much
of a leap from an approximately 1 in 500 chance of not developing
symptoms for more than 12 years to ``a significant number of people
who test positive never developing AIDS''.

\subsection{Fox hunting example}

We also considered an argument about fox hunting. Steps 1 to 4 of
argument evaluation are covered in the slides. But are the premises
true? The first premise is not really a scientific claim because it is
prescriptive rather than descriptive, as covered in the last
lecture. The second premise is too vague to investigate, because of
the term ``unnecessary''. What defines whether an event is
unnecessary? However, the related claim ``fox hunting causes
suffering'' is a claim for which the evidence can potentially examined
be scientifically --- if one were satisfied that animal suffering is
something that can be measured (which it probably is).

\subsection{Ethymemes}

This refers to a situation, undesirable but quite common even in
scientific writing, where one or more of the premises are missing and
have to be inferred by the reader.

\subsection{Using the flowchart in your writing}

In the end, it all comes back to the flowchart. You should also see
this as the receipe for constructing good arguments. One of the most
common problems in student essays and reports is that one or more of
these steps are missing in the arguments presented. Most commonly, the
conclusion is given, but the premises are not. Or, if they are, the
relationship between premise and conclusion is not clarified. Or the
truth of the premises is not examined.

\vspace{12pt}

\tiny
This work is licensed under a Creative Commons Attribution-ShareAlike
4.0 International Licence. Last update: \today

\end{document}
%%% Local Variables:
%%% mode: latex
%%% TeX-master: t
%%% End:
