\documentclass[12pt]{article}
\usepackage[a4paper,left=2.25cm,right=2.25cm,top=2cm,bottom=2.5cm]{geometry}
\usepackage{hyperref} % Handle web addresses
\usepackage{graphicx} % Handle graphics

\begin{document}
	\title{Science, truth, and honesty}
	\date{\today}
	\author{Andy J. Wills}
	\maketitle

\section{Topics}

This lecture series starts by considering two fundamental questions
--- ``what is truth?'' and ``what is science?''. We start with the
nature of truth.

\section{Truth as a property of claims}

It's not possible to get anywhere if one considers truth as some kind
of general ineffable property of the universe. Truth is, at heart, one
property of a claim. Someone makes a claim and someone (the same
person, or someone else) attributes the property ``true'' to it. Or
the property ``not true'' (false). It is also possible to attribute
some level of belief, perhaps using probability. Finally, it is
possible to ``opt out'' and say you have insufficient information to
make a sensible judgement.

Note that the last two statements on the slide are about the limits of
our knowledge, not (necessarily) about the limits of truth. The last
two statements are either true or false, it's just you don't know the
answer.

\section{Subjective and objective claims}

Truth is a property of claims, but there are different types of claims
--- a recurring theme in this lecture. The first distinction we need
to make is between subjective and objective claims.

Subjective claims are those whose truth differs for different
people. So, for example, the truth of the statement ``My favourite
colour is blue'' differs depending upon who says it. Objective claims
are those whose truth value is not affected by who says
it. ``Charlotte Smith's favourite colour is blue''. This is true, or
false, irrespective of whether I say it, you say it, or Charlotte
Smith says it.

Note that the term ``subjective'' is amongst the most widely mis-used,
particularly when referring to psychology. For example, the claim ``I
think chocolate tastes better than cabbage'' is subjective, but the
claim ``chocolate tastes better than cabbage'' is objective, and is
amenable to scientific test (for example, taste ratings taken over a
representative sample of individuals).

\section{Vague claims are subjective}

Critically, a claim is not objective if it is too vague to be
tested. For example, ``smoking is wrong'' - not objective because
``wrong'' has a number of different meanings.

\section{What is truth?}

When one asks ``what is truth?'', one generally is asking what it
means to say that an objective claim is true. One possibility is to
avoid the question by asserting that there are no objective claims --
all claims are subjective and hence the truth value of all claims
depends on who is speaking them.

This position about truth is called relativism. It's not an uncommon
view, even amongst academics, but it's pretty problematic.

First, any definition of relativism is self-defeating. For example,
relativism is the belief that ``all truth is relative''. This claim
is, in itself, universal and so proves that not all truth is relative!
The same problem exists for expressions such as ``No one can say what
is right or wrong''.

Another real problem with relativism is that it seems to preclude the
possibility of being wrong. If truth is simply what you believe, then
your beliefs cannot be in error.

One can avoid these paradoxes by saying, for example ``some truth is
relative''. This seems undeniably true, as we discussed under the
issue of subjective claims. It still leaves us with the question -
what does it mean to say an objective claim is true?

\section{Truth, correspondence, and coherence (2 slides)}

The answer is really quite prosaic. Truth can be defined both in terms
of correspondence and coherence. Anyone who has ever watched a crime
programme on TV knows this.

Correspondence Theory states that Truth is determined by the
correspondence between what is claimed and what is observed. An murder
suspect claims to have been at home between 10pm and midnight, but he
was seen by a reliable witness some miles from his house at that
time. The murder suspect's objective claim is judged to be false
because it does not correspond to observations.

Coherence Theory states that Truth is determined by the coherence
between new claims and other justified claims. The Defence claims that
the apparent murder victim actually committed suicide. A number of
Prosecution witnesses testify that the victim had no history of
depression or other mental illness, and appeared happy in the days and
weeks leading up the attack. The Defence's claim seems unlikely to be
true, not on the basis of anyone having observed how the injuries were
inflicted, but on the lack of coherence between this claim and other
justified claims.

\section{Scientific claims}

Having established what truth is, we can turn to the question of what
science is. The question of what science is, is important for two
reasons. First, you are studying a science and hence you are expected
to operate within the definition of what a scientist is. Second,
science is not just a subject of study, it is also one approach to
life in general that has much to commend it.

There are two parts to answering this question ``what is science'' -
the nature of scientific claims, and the culture of modern science.

Scientific claims first. At its heart, science is the process of
making scientific claims and then attempting to determine whether
those claims are true or false. Scientific claims are objective rather
than subjective, see earlier. Science also limits, or should limit,
the claims that it makes to those whose truth or falsity can, at least
in principle, be clearly determined.

In general, this means scientific claims are descriptive rather than
prescriptive.

\section{Descriptive versus prescriptive claims (3 slides)}

Science's focus on descriptive claims does not exempt scientists from
having moral or ethical principles, or from wanting to change the
world in particular ways. But scientists choose to investigate the
issues inherent in prescriptive claims by converting them to
descriptive claims that are relevant to the prescriptive point.

For example, a scientist who wishes to make the claim ``abortion
should be made illegal'' looks for an descriptive claim that would
support the prescriptive claim. For example, if one wished to support
the outlawing of abortion, one might make the claim ``feotal pain
receptors have developed by eight weeks gestation'', in order to
support your prescriptive claim. If one wished to argue against that
prescriptive claims, you might then make the descriptive claim
``foetal pain receptors are not connected to the brain until at least
20 weeks''.

Similarly, in order to support the prescriptive claim about
drink-driving, you would make some kind of descriptive claim. For
example, ``Risk of car accidents doubles at 80mg/100ml blood alcohol
(UK drink-driving limit)''

So, science is not about avoiding societally difficult questions. It
is about making claims that can be reliably examined, in order to help
answer difficult questions.

\section{Absolute versus contextual claims}

To summarize, scientific claims are objective, and they are usually
descriptive. However, scientific claims are seldom absolute.

Absolute claims are invariant. They hold always. Their truth value is
not conditional on circumstances. They are not conditional on time or
place.

Relative claims (perhaps better described as contextual claims) hold
under a defined set of conditions. Scientists should define those
conditions, but do not always do this. For example, consider the
claim, ``the leadership positions that women occupy are less promising
than those of their male counterparts''. This is not intended to be an
absolute claim. It is, rather, a claim about a current state of
affairs. If this claim is true now, but no longer true in 20 years,
perhaps due to increased awareness of the issue, this does not
undermine the truth value of the original claim -- as long as it is
seen in the appropriate context.

In general, scientists wish to investigate claims that are, as far as
possible, context independent. In other words, as close to being
absolute claims as is achievable. Indeed, it is hard to think of
claims as scientific in the broader sense if they are too
context-specific.  For example:

``The leadership positions that women in 2003 in UK FTSE100 companies
occupy are less promising than those of their male counterparts''

Such a claim is technically scientific, but the truth being claimed is
so contextual that it is not possible to, for example, make any
further predictions.

\section{Scientific claims and truth}

Scientific claims are objective, descriptive, and relatively context
independent. They also have a number of other properties.First, some
scientific claims are true and others are false. A claim that is false
can still be scientific if it meets the other criteria.

\section{Observable, measurable states}

Second, scientific claims are based on observable, measurable
states. Making sure your claim is descriptive rather than prescriptive
helps to make it observable and measurable, but doesn't always get you
all the way there. For example, consider the claim ``impulsive people
are more likely to be criminals''. Being a criminal is a state that is
observable and measurable in a number of relatively uncontroversial
ways. For example, if you have been convicted of a crime then you are
a criminal in an observable and measurable way.

Impulsivity is a fairly vague concept that has to be somehow
translated into something directly measurable. One of the
contributions of modern psychology has been to develop explicit
measures that can be clearly assessed for reliability and validity. In
the case of impulsivity, there is a 30-item standard questionnaire
called the Barratt Impulsivity Scale (BIS). So, our claim becomes:

``People with one or more criminal convictions score higher on the BIS
than people without a conviction''

\section{Independent replication}

Third, scientific claims must be expressed in such a way that they
permit independent replication. For example, claiming that Willsian
Therapy reduces depression is not a scientific claim if Wills is the
only person who can perform Willsian Therapy.

\section{Scientific claims}

Finally, scientific claims have to be falsifiable. This basically
means that it must be possible to imagine an outcome of an experiment
or other study that would lead to the conlcusion the claim was
false. If this is not possible, the claim is unfalsifiable and hence
unscientific. Note the difference between falsifiable (one can imagine
a set of circumstances in which it could be shown to be wrong), and
falsified (one has demonstrated that it is wrong). Good, productive,
science consists of the collection of claims that are falsifiable but
not (yet) falsified.

The slide contains a summary of the properties of falsifiable claims.

\section{Science and dishonesty (3 slides)}

To summarize, science is about making scientific claims, and we've
looked at what makes a claim scientific (or unscientific). But
science, like all human endeavours, is about more than this. It is
also about being part of a culture that holds certain behaviours and
processes in high regard. Trangressing these rules leads to the normal
cultural sanctions --- isolation, exclusion and, sometimes,
expulsion. In some cases, these cultural aspects represent A WAY of
doing science, rather than THE ONLY WAY it could be done.
		
The first cultural norm of modern science is honesty. This is most
easily defined by its opposite --- dishonesty. It is dishonest to
partially report your results if your intention is to obscure results
that are inconsistent with your claim. It is dishonest to choose a
form of data analysis because it gives you the best p-value, rather
than because it is the method best suited to your data.  It is
dishonest (if you are using standard statistical methods) to keep
testing until p $<$ .05. It is dishonest to publish the same data more
than once without acknowledging you have done this. And it is
dishonest to say you collected some data when actually you just made
it up!
		
The reason honesty is such an important cultural norm in science is
that it gets in the way of reliably evaluating claims, and wastes
peoples' time. Science is a highly social cooperative activity.

In the lecture, I considered the cases of Diderik Stapel, and Dirk
Smeesters. These are pretty obvious cases of unacceptable
dishonesty. I then went on to discuss Simmons et al.'s apparent
demonstration that listening to music about old age makes you
physically younger, as measured by your date of birth. This is of
course impossible --- and the authors realise this --- but the
conclusion comes from data that was really collected, and in which the
analyses reported were really conducted, and conducted correctly. The
problem is that, for the purposes of illustration, the authors engaged
in a number of other practices I listed above as dishonest (first
three items). Part of the problem is that some psychologists do not
consider these practices dishonest --- but this is changing.

\section{Science and skepticism (4 slides)}

Another important cultural norm in science is skepticism. Scientists
should welcome it when others are skeptical about their claims,
because such skepticism can lead to close examination of the evidence,
which in turn advances our understanding. Essentially, being
scientific is about soliciting and welcoming criticism. This is
something that many of us, including me, still find emotionally quite
difficult.

One recent development in skepticism is the increasing emphasis on not
only scientific claims permitting independent replication, but the
importance that such independent replication is actually
conducted. Recent meta-analyses suggest that a half to two-thirds of
results in psychology fail to replicate. In the lecture I discussed
two case studies (Ap Dijksterhuis, John Bargh).

I deliberately picked social psychology examples, because it is here
the results are worst. In social psychology, you should assume it
won't replicate unless there is evidence that it does. In cognitive
psychology, it's 50:50 --- your first question should still be ``has
this been independently repilcated?'' Ask this in all your lectures,
all your tutorials, all your essays. If it hasn't been independently
replicated, don't believe it!

\section{Science and publication}

The next cultural norm is publication. Science is highly social and
cooperative. Collecting data, and evaluating claims, and then keeping
the results of that evaluation to yourself is generally frowned
upon. Some scientists would even go so far as to say it is unethical
to not publish your work, although this is a minority view.

\section{Science and Open Access publication}

There are two important recent developments that have emerged from
this belief in the importance of publication.

The first is a realization that the traditional method of academic
publishing is getting in the way of the reasons we should be
publishing in the first place. Generally, scientists publish in
scientific journals. Subscriptions to these journals cost thousands of
pounds a year, and are generally only held in university libraries,
which are often not open to the public.

The solution is Open Access publishing. This means making your
publications freely available at no charge. There are a number of ways
to achieve this. For example, every scientific paper I have ever
written is available for free download at \url{www.willslab.org.uk}.

\section{Science and Open Access raw data}

Another recent development related to publication is the realization
that the internet now permits scientists to make not only their
publications, but also their raw data, freely, publicly, and
PERMANENTLY available. This is really incredibly important. It is one
thing to run a study again because you want to independently replicate
it. It is quite another to HAVE to run it again because the original
author didn't analyze the data properly and the raw data no longer
exists.

\section{Science and peer review}

A third, critical, aspect of the culture of modern science is the
process of peer review. This refers to the fact that your publications
should be scrutinized by your peers --- people who are experts in the
particular sub-field in which you are working. Peer review, despite
some criticism, remains a critical part of the scientific
process. Generally speaking, scientists (and others) would like to be
able to assume that publications have passed some basic quality
control checks. Authors would also like, in most cases, others to
believe what they are saying, rather than discard it because the work
contains some obvious flaw or is expressed unclearly. Peer review can
help with this.

\section{Peer-review process (2 slides)}

Traditionally, peer review proceeds as follows. The scientist writes a
paper and submits it to a journal. After some time (3-4 months in
psychology), the paper is returned with reviews from 2-3 experts in
the field. The journal editor tells the author either that the paper
has been accepted as is (very unlikely), or that it might be accepted
if the reviewers' comments can be addressed, or that they are not
interested -- please don't send it again. Resubmission of a revised
paper is typically followed by another set of reviews. Eventually, if
the editor agrees, the paper is published. The process can sometimes
take more than 2 years.

Peer review is traditionally anonymous and closed. It's anonymous in
the sense that you don't know the identity of the people reviewing
your work. It's closed in the sense that the reviews you receive are
confidential and cannot be published.

A hot topic at the moment is whether anonymous closed review tends to
lead to unscientific behaviour. As a reviewer, your identity is a
secret and your review will never be published. It is human nature to
be more critical of work that is counter to your own views than work
that is consistent with it. The cloak of anonymity and confidentiality
seems likely to make this worse. If reviews were published, along with
the reviewers' name, at the same time as the paper, reviewers would
presumably be rather more careful to limit their review to objective
unbiased comments.

The argument on the other side is that open named reviews would be
largely useless because people would be unwilling to give them, or
would not give their honest opinion for fear of retaliation. The
presence of good journals that use open named reviews as part of their
process suggests this fear is unfounded. For example, Behavioral and
Brain Sciences publishes papers alongside several invited open peer
reviews, and an authors' response to those reviews. BBS also runs a
closed review process before this as a quality-control check.

\section{Culture of modern science (and future directions)}

This is the final slide of the lecture, and summarizes the culture of
modern science.

\vspace{12pt}

\tiny
This work is licensed under a Creative Commons Attribution-ShareAlike
4.0 International Licence. Last update: \today


\end{document}
%%% Local Variables:
%%% mode: latex
%%% TeX-master: t
%%% End:
